\documentclass{article}
\usepackage[utf8]{inputenc}

\title{Quiz 2}
\author{Allen Williams }
\date{January 31st, 2018}

\usepackage{amsthm}
\usepackage{amsmath}
\usepackage{amssymb}
\usepackage{scrextend}
\newtheorem*{Theorem}{Theorem}
\newtheorem*{Axiom}{Axiom}
\newtheorem{Problem}{Problem}
\renewcommand\qedsymbol{QED}

\binoppenalty=\maxdimen
\relpenalty=\maxdimen

\begin{document}
\maketitle

\begin{Problem}
 i) Find the gcd of 203 and 91 using the Euclidean Algorithm. \\
 \begin{align*}
     203&=91\cdot 2+21 \\
     91&=21\cdot 4 + 7 \\
     21&=7\cdot 3 + 0 \\
 \end{align*}
 So by the Euclidean Algorithm, gcd(203, 91)=7.  Working backwards 7 can be written as a linear combination of 203 and 91. \\
 \begin{align*}
     7&=91-21\cdot 4 \\
     7&=91-4*(203-2\cdot 91)=9\cdot91+(-4)\cdot23
 \end{align*}
So gcd(203, 91) = 7 = $9\cdot91+(-4)\cdot23$ \\
ii) Find the gcd of 21 and 8 using the Euclidean Algorithm. \\
\begin{align*}
    21&=8\cdot 2+5 \\
    8&=5\cdot1+3 \\
    5&=3\cdot1+2 \\
    3&=2\cdot1+1 \\
    2&=1\cdot2+0 \\
\end{align*}
So by the Euclidean Algorithm, gcd(21, 8)=1.  Working backwards 1 can be written as a linear combination of 21 and 8. \\
\begin{align*}
    1&=3-2 \\
    1&=3-(5-3)=3\cdot2-5 \\
    1&=2(8-5)-5=2\cdot8-3\cdot5 \\
    1&=2\cdot8-3(21-2\cdot8)=8\cdot8-3\cdot21 \\
\end{align*}
So gcd(21, 8) = 1 = $8\cdot8-3\cdot21$
\end{Problem}
\begin{Problem}
Prove that for $n\geq1$, $2+2^2+...+2^n=2^{n+1}-2$ \\
The proof is by induction.  Let $P(n)$ denote the statement "$2+2^2+...+2^n=2^{n+1}-2$". Clearly $2^1=2(2-1)=2^2-2^1=2^{1+1}-2$ so $P(1)$ is true establishing the basis of induction.  Assume $P(k)$ is true for some $k\in\mathbb{N}$, that is \\
\begin{align*}
    2+2^2+...+2^k&=2^{k+1}-2\text{  so,} \\
    2+2^2+...+2^k+2^{k+1}&=2^{k+1}-2+2^{k+1} \\
    &=2\cdot2^{k+1}-2 \\
    &=2^{(k+1)+1}-2 \\
\end{align*}
So $P(k+1)$ is true and by the first principle of mathematical induction $P(n)$ is true for all $n\in\mathbb{N}$.
\end{Problem}
\begin{Problem}
Prove that for all $n\geq 1$, $8^n-3^n$ is divisible by 5. \\
The proof is by induction.  Let $P(n)$ denote the statement "$8^n-3^n$ is divisible by 5."  $8^1-3^1=5=5\cdot1$ so $P(1)$ is true, establishing the basis of induction.  Assume $P(k)$ is true for some $k\in\mathbb{N}$, that is \\
\begin{align*}
    8^k-3^k&=5\cdot j\text{ for some }j\in\mathbb{Z}\text{ so,} \\
    8(8^k-3^k)&=8\cdot 5j \\
    8\cdot8^k-8\cdot3^k&=8\cdot 5j \\
    (8\cdot 8^k-3\cdot 3^k)-5\cdot 3^k&=5\cdot 8j \\
    8^{k+1}-3^{k+1}&=5\cdot 8j+5\cdot 3^k \\
    8^{k+1}-3^{k+1}&=5(8j+3^k) \\
\end{align*}
Since $8j$ and $3^k\in\mathbb{Z}$, $8^{k+1}-3^{k+1}$ is divisible by 5 so by the first principle of mathematical induction, $P(n)$ is true for all $n\in\mathbb{N}$.
\end{Problem}
\pagebreak
\begin{Problem}
Prove that $\frac{1}{2}+\frac{1}{6}+...+\frac{1}{n(n+1)}=\frac{n}{n+1}$ for all $n\in\mathbb{N}$. \\
The proof is by induction.  Let $P(n)$ denote the statement "$\frac{1}{2}+\frac{1}{6}+...+\frac{1}{n(n+1)}=\frac{n}{n+1}$". $\frac{1}{1(1+1)}=\frac{1}{2}=\frac{1}{1+1}$ so $P(1)$ is true, establishing the basis of induction.  Assume $P(k)$ is true for some $k\in\mathbb{N}$ that is, \\
\begin{align*}
    \frac{1}{2}+\frac{1}{6}+...+\frac{1}{n(n+1)}&=\frac{n}{n+1}\text{ then,} \\
    \frac{1}{2}+\frac{1}{6}+...+\frac{1}{n(n+1)}+\frac{1}{(n+1)((n+1)+1)}&=\frac{n}{n+1}+\frac{1}{(n+1)((n+1)+1)} \\
    &=\frac{n}{n+1}+\frac{1}{(n+1)(n+2)} \\
    &=\frac{n(n+2)+1}{(n+2)(n+1)} \\
    &=\frac{n^2+2n+1}{(n+1)(n+2)} \\
    &=\frac{(n+1)^2}{(n+1)(n+2)} \\
    &=\frac{n+1}{(n+1)+1} \\
\end{align*}
So $P(k+1)$ is true and by the first principle of mathematical induction, $P(n)$ is true for all $n\in\mathbb{N}$.
\end{Problem}
\begin{Problem}
Demonstrate that $\sqrt{7}$ cannot be a rational number. \\
Assume for contradiction that $\sqrt{7}$ is rational, that is $\sqrt{7}=\frac{p}{q}$  for $p,q\in\mathbb{Z}$ with $q\neq0$ and $gcd(p,q)=1$.  Then $7=\frac{p^2}{q^2}$ so $7q^2=p^2$ so $7\mid p^2$ and since 7 is a prime number $7\mid p$, that is $p=7j$ for some $j\in\mathbb{Z}$.  Then $7q^2=(7j)^2$ so $7q^2=49j^2$ so $q^2=7j^2$ meaning $7\mid q^2$ and again since 7 is a prime number, $7\mid q$. Then 7 is a common divisor of $p$ and $q$ which contradicts the assumption that $1=gcd(p,q)$, so $\sqrt{7}$ cannot be a rational number.
\end{Problem}
\begin{Problem}
Let $a,b,c\in\mathbb{Z}$.  Prove that if $gcd(a,b)=1$ and $a\mid bc$, then $a\mid c$. \\
Let $a,b,c\in\mathbb{Z}$ suppose that $gcd(a,b)=1$ and $a\mid b\cdot c$.  Then $bc=ak$ for some $k\in\mathbb{Z}$.  Also by Bezout's identity there exist $r,s\in\mathbb{Z}$ such that $ar+bs=1$.  Then $car+(bc)s=c$, so $car+aks=c$, so $a(cr+ks)=c$.  Since c,r,k and s are integers, $a\mid c$.
\end{Problem}
\begin{Problem}
Deleted 1/29/2018
\end{Problem}
\end{document}