\documentclass{article}
\usepackage[utf8]{inputenc}

\title{Quiz 4}
\author{Allen Williams }
\date{February 14th, 2018}

\usepackage{booktabs}
\usepackage{dcolumn}
\usepackage{amsthm}
\usepackage{amsmath}
\usepackage{amssymb}
\usepackage{scrextend}
\newtheorem*{Theorem}{Theorem}
\newtheorem*{Axiom}{Axiom}
\newtheorem{Problem}{Problem}
\renewcommand\qedsymbol{QED}

\makeatletter
\newcommand{\thickhline}{%
    \noalign {\ifnum 0=`}\fi \hrule height 1pt
    \futurelet \reserved@a \@xhline
}
\newcolumntype{"}{@{\hskip\tabcolsep\vrule width 1pt\hskip\tabcolsep}}
\makeatother


%\binoppenalty=\maxdimen
%\relpenalty=\maxdimen

\begin{document}

\maketitle

\begin{enumerate}
\item Prove that $G=\{a+b\sqrt{3}\mid a,b\in\mathbb{Q}, a,b\text{ not both zero}\}$ is a subgroup of $\mathbb{R}^*$, the multiplicative group of non-zero real numbers.  Let $x,y\in G$ then $x=a+b\sqrt{3}$ for some $a,b\in\mathbb{Q}$ and $y=a'+b'\sqrt{3}$ for some $a',b'\in\mathbb{Q}$.  Then $y^{-1}=\frac{1}{a'+b'\sqrt{3}}$, so $x\cdot y^{-1}=\frac{a+b\sqrt{3}}{a'+b'\sqrt{3}}=\frac{a+b\sqrt{3}}{a'+b'\sqrt{3}}\cdot \frac{a'-b'\sqrt{3}}{a'-b'\sqrt{3}}\newline=\frac{aa'-3bb'}{(a')^2-3(b')^2}+\frac{a'b-ab'}{(a')^2-3(b')^2}\sqrt{3}$.  So when $x,y\in G$, $x\cdot y^{-1}\in G$ which is a necessary and sufficient condition for $G$ being a subgroup of $\mathbb{R}^*$.

\item Prove the $SL_2(\mathbb{Z})$ is a subgroup of $SL_2(\mathbb{R})$.  Clearly $SL_2(\mathbb{Z})$ is a subset of $SL_2(\mathbb{R})$ since $\mathbb{Z}$ is a subset of $\mathbb{R}$.  Let $X=\left( \begin{array}{cc} a & b\\ c & d \end{array} \right)$ and $Y=\left( \begin{array}{cc} a' & b'\\ c' & d' \end{array} \right)$ be elements of $SL_2(\mathbb{Z})$, that is $a,b,c,d,a',b',c',d'\in \mathbb{Z}$ and $\det(X)=\det(Y)=1$.  Then $XY^{-1}=\frac{1}{\det(Y)}X\cdot adj(Y)=\left( \begin{array}{cc} a & b\\ c & d \end{array}\right)\left( \begin{array}{cc} d' & -b'\\ -c' & a' \end{array}\right)=\left( \begin{array}{cc} ad'-bc' & a'b-ab' \\ cd'-c'd & a'd-b'c \end{array}\right)$.  Since the entries in $X$ and $Y^{-1}$ are integers, the entries in $XY^{-1}$ are also integers.  Also since $\det(Y)=1$, $\det(Y^{-1})=\frac{1}{1}=1$, so $\det(XY^{-1})=\det(X)\det(Y^{-1})=1\cdot1=1$, so $XY^{-1}\in SL_2(\mathbb{Z})$ when $X,Y\in SL_2(\mathbb{Z})$, meaning $SL_2(\mathbb{Z})$ is a subgroup of $SL_2(\mathbb{R})$.

\item $<\mathbb{Q},+>$ the additive group of rational numbers is not cyclic.  First note that if $\mathbb{Q}$ were generated by one of its elements, that element could not be 0, since 0 is the identity element in $<\mathbb{Q}, +>$ so $<0>=\{0\}$.  Now Assume for contradiction that $\mathbb{Q}=<a>$ for some $a\in\mathbb{Q}$ with $a\neq 0$, that is for all $q\in \mathbb{Q}$, there exists an integer $k$, such that $q=ka$.  Since $a$ is itself rational, $\frac{a}{2}$ must also be rational.  Consider $q=\frac{a}{2}$, then $a=2ka$.  Since $a\neq 0$, $2k=1$ so $k=\frac{1}{2}$, which contradicts the fact that $k$ is an integer, so $<\mathbb{Q},+>$ cannot be cyclic.

\item What is the order of $[4]$ in $U(21)$?  Find the subgroup of $U(21)$ generated by $[4]$.  The order of $[4]$ in $U(21)$ is 3 since $[4]^3=[4*4*4]=[64]=[1]$.  $<[4]>=\{[4],[16],[1]\}$.

\item $U(5)$ is a cyclic group and its generators are $[2]$ and $[3]$.  $U(5)$ consists of the elements $\{[1],[2],[3],[4]\}$.  Also $[2]\cdot[3]=[6]=[1]$ so $[2]=[3]^{-1}$  so the subgroups $<[2]>$ and $<[3]>$ will be the same.  \begin{align*}
    [2]^1&=[2] \\
    [2]^2&=[4] \\
    [2]^3=[8]&=[3] \\
    [2]^4=[16]&=[1]
\end{align*}
So $<[2]>=<[3]>=U(5)$. $[1]$ is the identity element in $U(5)$ so it cannot generate $U(5)$, and $[4]$ is its own inverse so it cannot generate $U(5)$ so $[2]$ and $[3]$ are the only generators for $U(5)$.  Since $[4]$ is its own inverse, $<[4]>=\{[4],[1]\}$ which is a non-trivial subgroup of $U(5)$.

\item Find all non-trivial cyclic subgroups of $<\mathbb{Z}_8,+>$.  $[0]$ is the identity on $<\mathbb{Z}_8,+>$ so $<[0]>$ is the trivial group.  Also since $[1]=[7]^{-1}, [2]=[6]^{-1},$ and $[3]=[5]^{-1}$, and an element generates the same cyclic subgroup as its inverse, it is enough to find $<[1]>,<[2]>,<[3]>,$ and $<[4]>$.  \begin{align*}
    [1]&=[1] \\
    [1]+[1]&=[2] \\
    [1]+[1]+[1]&=[3] \\
    [1]+[1]+[1]+[1]&=[4] \\
    [1]+[1]+[1]+[1]+[1]&=[5] \\
    [1]+[1]+[1]+[1]+[1]+[1]&=[6] \\
    [1]+[1]+[1]+[1]+[1]+[1]+[1]&=[7] \\
    [1]+[1]+[1]+[1]+[1]+[1]+[1]+[1]=[8]&=[0]
\end{align*}
So $<[1]>=\{[0],[1],[2],[3],[4],[5],[6],[7]\}$
\begin{align*}
    [2]&=[2] \\
    [2]+[2]&=[4] \\
    [2]+[2]+[2]&=[6]\\
    [2]+[2]+[2]+[2]=[8]&=[0]
\end{align*}
So $<[2]>=\{[0],[2],[4],[6]\}$
\begin{align*}
    [3]&=[3] \\
    [3]+[3]&=[6] \\
    [3]+[3]+[3]=[9]&=[1] \\
    [3]+[3]+[3]+[3]=[12]&=[4] \\
    [3]+[3]+[3]+[3]+[3]=[15]&=[7] \\
    [3]+[3]+[3]+[3]+[3]+[3]=[18]&=[2] \\
    [3]+[3]+[3]+[3]+[3]+[3]+[3]=[21]&=[5] \\
    [3]+[3]+[3]+[3]+[3]+[3]+[3]+[3]=[24]&=[0]
\end{align*}
So $<[3]>=\{[0],[1],[2],[3],[4],[5],[6],[7]\}$
\begin{align*}
    [4]&=[4] \\
    [4]+[4]=[8]&=[0]
\end{align*}
So $<[4]>=\{[4],[0]\}$

\end{enumerate}

\end{document}
