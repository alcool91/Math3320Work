\documentclass{amsart}
\usepackage[utf8]{inputenc}

\title{Quiz 4}
\author{Allen Williams }
\date{March 14th, 2018}

\usepackage{booktabs}
\usepackage{dcolumn}
\usepackage{amsthm}
\usepackage{amsmath}
\usepackage{amssymb}
\usepackage{scrextend}
\newtheorem*{Theorem}{Theorem}
\newtheorem*{Axiom}{Axiom}
\newtheorem{Problem}{Problem}
\renewcommand\qedsymbol{QED}

\makeatletter
\newcommand{\thickhline}{%
    \noalign {\ifnum 0=`}\fi \hrule height 1pt
    \futurelet \reserved@a \@xhline
}
\newcolumntype{"}{@{\hskip\tabcolsep\vrule width 1pt\hskip\tabcolsep}}
\makeatother


%\binoppenalty=\maxdimen
%\relpenalty=\maxdimen

\begin{document}

\maketitle

\begin{enumerate}
    \item Show that $A_{10}$ contains an element of order 15.
    $A_{10}$ is the subgroup of even permutations in $S_{10}$.  Consider the element $(12345)(678)$ in $S_{10}$.  The left cycle has order 5 and the right cycle has order 3, and the two cycles are disjoint so the order of the permutation given by the product of the two cycles is $LCM(5,3)=15$.  Also the 5-cycle can be written as the product of 4 transpositions and the 3-cycle can be written as the product of 2 transpositions.  Since 4+2=6 and any element that can be written as the product of an even number of transpositions is an even permutation, $(12345)(678)$ is an element in $A_{10}$ of order 15.
    \item Does $A_8$ contain an element of order 26?
    $A_8$ does not contain an element of order 26.  An element in $A_8$ can be written as a product of disjoint cycles.  Since the order of the product of disjoint cycles is the LCM of the length of the cycles, possible cycle lengths are 26 or 13 and 2.  A cycle of length 26 can be written as a product of 25 transpositions and is therefore an odd permutation and not in $A_8$.  Similarly, a product of a length 13 cycle and a length 2 cycle can be written as a product of 13 transpositions and is not in $A_8$.
    \item Let $\sigma$ be a cycle of odd length, then $\sigma=(123...2k+1)$, so $\sigma^2=(123...2k+1)(123...2k+1)=(135...(2k+1)246...2k)$ so $\sigma^2$ is a cycle.
    \item $D_{10}$ contains the rigid motions of a regular 10-gon.  It is generated by $r$ and $s$ where $r$ has order 10 and $s$ has order 2.  Consider an arbitrary element $s^pr^k$ where $p\in\{0,1\}$ and $0\leq k\leq 9$.  Since the group is generated by $r$ and $s$ if an element commutes with $r$ and $s$ it will commute with any element of the group.  It is true that $sr^ksr^k=e$ so $sr^k=(sr^k)^{-1}=r^{-k}s$ So if $r^k=r^-k$, $r^k$ and $s$ will commute.  This is true precisely for $k=\frac{n}{2}$ or $r^5$.  Since $\langle r \rangle$ is a cyclic subgroup its elements commute with each other, so $r^5$ also commutes with $r$.  Clearly no other non-identity element can have this property so the center of $D_{10}$ is $\{e,r^5\}$.  In $D_7$ the situation is even more grim since there is no element $r^k$ which is its own inverse in $D_7$ since the order of $\langle r \rangle$ is odd here, so the center of $D_7=\{e\}$. 
    \item Let $\alpha, \beta\in S_n$ and define $(\alpha,\beta)\in R$ if there exists a $\sigma\in S_n$ such that $\sigma \alpha \sigma^{-1}=\beta$.\\
    Consider $(\alpha,\alpha)$.  Clearly $e\alpha e^{-1}=\alpha$ so $(\alpha,\alpha)\in R$ for all $\alpha \in S_n$, so $R$ is reflexive. \\
    Suppose $(\alpha, \beta)\in R$ then there exists an element $\sigma$ such that $\sigma \alpha \sigma^{-1}=\beta$.  By multiplying on the right by $\sigma$ and the left by $\sigma^{-1}$ observe $\alpha=\sigma^{-1}\beta \sigma=\sigma^{-1}\beta (\sigma^{-1})^{-1}$, so $(\beta, \alpha)\in R$ so $R$ is symmetric.  Now suppose $(\alpha, \beta)\in R$ and $(\beta, \xi)\in R$, then there exist $\sigma_1$ and $\sigma_2$ such that $\sigma_1 \alpha \sigma_1^{-1}=\beta$ and $\sigma_2 \beta \sigma_2^{-1}=\xi$.  Solving the second equation for $\beta$ and substituting gives $\sigma_1 \alpha \sigma_1^{-1}=\sigma_2^{-1} \xi \sigma_2^$, or $(\sigma_2\sigma_1)\alpha (\sigma_2\sigma_1)^{-1}=\xi$.  Since $\sigma_1, \sigma_2\in S_n$, $\sigma_1\sigma_2\in S_n$ so $(\alpha, \xi)\in R$ so R is transitive.\\
    The partition of $S_3$ given by $R$ is $\{\{(123),(132)\},\{(1)\},\{(12),(13),(23)\}\}$
    \item For any $\alpha \in S_n$ $\alpha^{-1}$ has the same parity as $\alpha$ since $\alpha \alpha^{-1}=(1)$ and $(1)$ is an even permutation.  Then simply writing $\alpha^{-1}\beta^{-1}\alpha \beta$ as a product of transpositions we see that we have either 2 even numbers and 2 odd numbers, 4 even numbers, or 4 odd numbers.  Sums of even numbers are even so if $\alpha, \beta$ are both even clearly $\alpha^{-1} \beta^{-1} \alpha \beta$ is even.  Sums of 2 odd numbers are even as well so if $\alpha, \beta$ have opposite parity then the number of transpositions in the two even permutations is even and the number of permutations in the two odd permutations is even, so overall $\alpha^{-1}\beta^{-1}\alpha\beta$ is even.  Similarly when $\alpha, \beta$ are both odd, 4 odd numbers of transpositions add together to an even number of transpositions, so $\alpha^{-1}\beta^{-1}\alpha\beta$ is even.
    \item List the cosets of $\langle 3 \rangle$ in $U(8)$. \\
    First note $U(8)=\{1,3,5,7\}$ and $\langle 3 \rangle=\{1,3\}.$  Then \begin{align*}
        1\cdot \langle 3 \rangle&=\{1,3\} \\
        3\cdot \langle 3 \rangle&=\{3,1\} \\
        5\cdot \langle 3 \rangle&=\{5,15\}=\{5,7\} \\
        7\cdot \langle 3 \rangle&=\{7,21\}=\{7,5\} \\
    \end{align*}
    \item There are 5 cosets of $\langle 5\mathbb{Z}, + \rangle$ in $\langle \mathbb{Z},+ \rangle$ which are $\{5z\mid z\in\mathbb{Z}\}$, $\{5z+1\mid z\in\mathbb{Z}\}$, $\{5z+2\mid z\in\mathbb{Z}\}$, $\{5z+3\mid z\in\mathbb{Z}\}$, and $\{5z+4\mid z\in\mathbb{Z}\}$.  Clearly the next coset would overlap the first.
    \item Since the order of $S_5=5!=120$ and the order of $\langle g \rangle=4$ (clearly $g^4=e$), by Lagrange's Theorem $[G:H]=\frac{120}{4}=30$.
    \item The index of $SL_2(\mathbb{R}$ in $GL_2(\mathbb{R})$ is infinite.  Consider left cosets $gH=\{gh\mid h\in SL_2(\mathbb{R})\}$.  Since $g\in GL_2(\mathbb{R}$, $det(g)\neq 0$ and $det(h)=1$ for all $h\in SL_2(\mathbb{R})$.  Since $det(h)=1$, $det(gh)=det(g)$.  Also the determinant is a surjective mapping from $GL_2(\mathbb{R})$ to $R^*$ since we can take an arbitrary $a\in R^*$ and observe $g_a=\left( \begin{array}{cc} 1 & 0 \\ 0 & a\end{array}\right)$ has determinant $a$.  Finally all the elements of a coset $g_aH$  have determinant $a$ and since $a$ is an arbitrary non-zero real number, there are an infinite number of these cosets.
\end{enumerate}

\end{document}
