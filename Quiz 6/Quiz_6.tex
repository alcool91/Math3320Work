\documentclass{amsart}
\usepackage[utf8]{inputenc}

\title{Quiz 6}
\author{Allen Williams }
\date{\today}

\usepackage{booktabs}
\usepackage{dcolumn}
\usepackage{amsthm}
\usepackage{amsmath}
\usepackage{amssymb}
\usepackage{scrextend}
\newtheorem*{Theorem}{Theorem}
\newtheorem*{Axiom}{Axiom}
\newtheorem{Problem}{Problem}
\renewcommand\qedsymbol{QED}

\makeatletter
\newcommand{\thickhline}{%
    \noalign {\ifnum 0=`}\fi \hrule height 1pt
    \futurelet \reserved@a \@xhline
}
\newcolumntype{"}{@{\hskip\tabcolsep\vrule width 1pt\hskip\tabcolsep}}
\makeatother


%\binoppenalty=\maxdimen
%\relpenalty=\maxdimen

\begin{document}

\maketitle

\begin{enumerate}
    \item $\mathbb{Z}_{24}$ is not isomorphic to $D_{12}$ since $\mathbb{Z}_{24}$ is cyclic and $D_{12}$ is not, also $\mathbb{Z}_{24}$ is Abelian and $D_{12}$ is not.  Isomorphisms take cyclic groups to cyclic groups and Abelian groups to Abelian groups so $\mathbb{Z}_{24}$ and $D_{12}$ cannot be isomorphic.
    \item $\langle \mathbb{Q},+\rangle$ cannot be isomorphic to $\langle \mathbb{Z},+\rangle$ since $\langle \mathbb{Z},+\rangle$ is cyclic but $\langle \mathbb{Q},+\rangle$ is not.
    \item $U(8)$ and $\mathbb{Z}_4$ are not isomorphic.  Every element in $U(8)$ is its own inverse, so none of them generates the group, but $\mathbb{Z}_4$ is cyclic and generated by the element $1$, so no isomorphism can exist between $\mathbb{Z}_4$ and $U(8)$.
    \item Let $\phi:G_1\to G_2$ and $\psi:G_2\to G_3$ be two isomorphisms.  $\phi^{-1}$ is bijective since $\phi$ is bijective.  Consider $\phi(x),\phi(y)\in G_2$.  $\phi^{-1}(\phi(x)\phi(y))=\phi^{-1}(\phi(xy))=xy=\phi^{-1}\phi(x)\phi^{-1}\phi(y)$, so $\phi^{-1}$ is an isomorphism.  Consider $\psi\circ\phi(xy)=\psi(\phi(xy)).$  Since $\phi$ is an isomorphism this equals $\psi(\phi(x)\phi(y))$ and since $\psi$ is an isomorphism it equals $\psi(\phi(x))\psi(\phi(y))=\psi\circ\phi(x)\psi\circ\phi(y)$, so $\psi\circ\phi$ is an isomorphism.
    
    Consider the set of groups and define $(G_1, G_2)\in R$ iff $G_1\equiv G_2$.  $(G_1, G_1)\in R$ for all groups since $\epsilon:G\to G$ given by $\epsilon(x)=x$ is clearly an isomorphism.  Syppose $(G_1, G_2)\in R$ then there exists a function $\phi:G_1\to G_2$ such that $\phi$ is an isomorphism.  Since $\phi$ is bijective it has an inverse, which was shown above to be an isomorphism.  So isomorphism is transitive, symmetric, and reflexive and forms an equivalence relation on the set of groups.
    \item $\Phi(a+bi)=(a-bi)$ is an isomorphism from $\mathbb{C}$ to $\mathbb{C}$.  Suppose $\Phi(a_1+b_1i)=\Phi(a_2+b_2i)$, then $a_1-b_1i=a_2-b_2i$.  These two complex numbers being equal implies $a_1=a_2$ and $b_1=b_2$, which implies $a_1+b_1i=a_2+b_2i$, so $\Phi$ is injective.
    
    Let $y\in\mathbb{C}$ be arbitrary, that is $y=a+bi$ for some $a,b\in\mathbb{R}$ then $\Psi(a-bi)=a+bi=y$, so $\Phi$ is surjective.
    \begin{align*}
    \Phi(a_1+b_1i+a_2+b_2i&=\Psi((a_1+a_2)+(b_1+b_2)i) \\
    &=a_1+a_2-(b_1+b_2)i \\
    &=a_1-b_1i+a_2-b_2i \\
    &=\Phi(a_1+b_1i)+\Phi(a_2+b_2)i
    \end{align*}
    So $\Phi$ is an isomorphism from $\mathbb{C}$ to $\mathbb{C}$.
    
    \item $\Phi$ was shown to be bijective above.
    \begin{align*}
        \Phi((a_1+b_1i)(a_2+b_2i))&=\Phi(a_1a_2+a_1b_2i+a_2b_1i-b_1b_2) \\
        &=\Phi((a_1a_2-b_1b_2)+(a_1b_2+a_2b_1)i) \\
        &=a_1a_2-b_1b_2-(a_1b_2+a_2b_1)i \\
        &=a_1a_2-b_1b_2-a_1b_2i-a_2b_1i \\
        &=(a_1-b_1i)(a_2-b_2i) \\
        &=\Psi(a_1-b_1i)\Psi(a_2-b_2i)
    \end{align*}
    So $\Psi$ is an isomorphism from $\mathbb{C^*}$ to $\mathbb{C^*}$
    
    \item Let $G$ be a group and let $g\in G$.  Define $i_g:G\to G$ by $i_g(x)=gxg^{-1}$.
    
    Suppose $i_g(x_1)=i_g(x_2)$ then $gx_1g^{-1}=gx_2g^{-1}$.  Multiply on the left by $g^{-1}$ and the right by $g$ to get $x_1=x_2$ so $i_g$ is injective.
    
    Let $y\in G$ then $i_g(g^{-1}yg)=gg^{-1}ygg^{-1}=y$ so $i_g$ is surjective.
    
    $i_g(xy)=i_g(xey)=i_g(xg^{-1}gy)=gxg^{-1}gyg^{-1}=i_g(x)i_g(y)$
    so $i_g$ is an isomorphism from $G$ to $G$.
    
    \item let $\xi$ and $\phi$ be two inner automorphisms and consider $\xi\phi^{-1}(x)=\xi(\phi^{-1}(x))=\xi(g^{-1}xg)=hg^{-1}xgh^{-1}$ for $g,h\in G$.
    
    Suppose $hg^{-1}xgh^{-1}=hg^{-1}ygh^{-1}$ then multiply on the left by $h^{-1}g$ and the right by $hg^{-1}$ to get $x=y$ so $\xi\phi^{-1}$ is injective.
    
    Let $z\in G$ be arbitrary then $z=\xi\phi^{-1}(gh^{-1}zhg^{-1})$ so $\xi\phi^{-1}$ is surjective.
    
    $\xi\phi^{-1}(xy)=\xi\phi^{-1}(xgh^{-1}hg^{-1}y)=hg^{-1}xgh^{-1}hg^{-1}ygh^{-1}=\xi\phi^{-1}(x)\xi\phi^{-1}(y)$ So  $\xi\phi^{-1}\in Aut(G)$ so $Inn(G)$ is a subgroup of $Aut(G)$.
    
    \item let $G$ be a group and let $g\in G$.  Define $\lambda_g:G\to G$ and $\rho_g:G\to G$ such that $\lambda_g(x)=gx$ and $\rho_g(x)=xg^{-1}$.  Define $i_g=\rho_g\circ \lambda_g$, then $i_g=\rho_g(\lambda_g(x))=\rho_g(gx)=gxg^{-1}$ so $i_g\in Inn(G)$ which is a subgroup of $Aut(G)$ so $i_g\in $ $Aut(G)$
    
    Define $f:G\to S_G$ such that $f(g)=\rho_g$.  It must be shown that $f$ is an injective homomorphism from $G$ to $S_G$.  Suppose \begin{align*}
        f(g_1)&=f(g_2) \\
        \rho_g_1&=\rho_g_2 \\
        \rho_g_1(x)&=\rho_g_2(x) \\
        xg_1^{-1}&=xg_2^{-1} \\
        g_1^{-1}&=g_2^{-1} \\
        g_1&=g_2
    \end{align*}
    So f is injective.
    
    to see that $f(g_1g_2)=f(g_1)\circ f(g_2)$ observe that \begin{align*}
    f(g_1)\circ f(g_2)&=\rho_g_1 \circ \rho_g_2 \\
    &=\rho_g_1(\rho_g_2(x)) \\
    &=\rho_g_1(xg_2^{-1}) \\
    &=xg_2^{-1}g_1^{-1} \\
    &=x(g_1g_2)^{-1} \\
    &=\rho_{g_1g_2} \\
    &=f(g_1g_2)
    \end{align*}
    So $f$ is an injective homomorphism between $G$ and $S_G$ as required for Cayley's Theorem.
    
    \item Automorphisms take generators to generators.  Generators of $\mathbb{Z_6}$ are $1$ and $5$, so possible automorphisms are $\epsilon(1)=1$ or $f(1)=5$, which gives \begin{align*}
        f(2)=f(1)+f(1)=4 \\
        f(3)=f(2)+f(1)=3 \\
        f(4)=f(3)+f(1)=2 \\
        f(5)=f(4)+f(1)=1 \\
        f(0)=f(5)+f(1)=0
    \end{align*}
    Also $\mathbb{Z}$ has two generators, $1$ and $-1$ so $Aut(Z)$ consists of $\epsilon(x)=x$ and $\tau(x)=-x$.
\end{enumerate}
\end{document}
